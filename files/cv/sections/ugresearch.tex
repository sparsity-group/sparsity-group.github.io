\begin{rSection}{Undergraduate Research}

  \begin{tabularx}{\linewidth}{Xr}
  \mc{1}{l}{\bf Efficient Top-\boldmath$K$ Ranking from Noisy Pairwise Comparisons} & January 2017 -- Present \\
    \mcX{2}{Advisor: \href{\soheil}{Soheil {\sc Mohajer}}} \\
    \mcX{2}{University of Minnesota -- Department of Electrical \& Computer Engineering} \\
    \mcX{2}{\footnotesize Working on the design of an efficient top-$K$ rank
    aggregation algorithm that ranks items from noisy pairwise comparisons.
    This work is supported by a
    UROP award.
    } \\
    \mc{2}{c}{} \\ % just to get an empy row
    \mc{1}{l}{\bf MP3: A More Efficient Private Presence Protocol} & January
    2016 -- Present \\
    \mcX{2}{Advisor: \href{\hopper}{Nicholas J. {\sc Hopper}}} \\
    \mcX{2}{University of Minnesota -- Department of Computer Science \& Engineering} \\
    \mcX{2}{
      \footnotesize{Developing and implementing an efficient {\em privacy-preserving}
      cryptographic presence protocol. The notion of presence protocol is for one
      to advertise to their `buddies' if they are online. This is usually
      implemented such that the `central server' knows the entire graph structure
      of the social network. We propose MP3---the Minnesota Private Presence
      Protocol---an efficient presence protocol that leaks {\bf no} information
      about the social graph and maintains forward secrecy in the event of a
      compromise.
      This work was previously supported by a UROP
      award, and is currently supported by a UGRA appointment. An arXiv
      preprint of this work is currently available (see Publications).}
    } \\
    \mc{2}{c}{} \\ % just to get an empy row
    \mc{1}{l}{\bf Fault-Tolerant Ripple-Carry Adder using PTMR} & August 2014 -- February 2015 \\
    \mcX{2}{Advisors: \href{\chriskim}{Chris H. {\sc Kim}} \& \href{\parhi}{Keshab K. {\sc Parhi}}} \\
    \mcX{2}{University of Minnesota -- Department of Electrical \& Computer Engineering} \\
    \mcX{2}{
      \footnotesize{Developed new variant of triple modular redundancy, referred to as {\it partial triple modular redundancy} (PTMR), to harden arithmetic circuits from soft errors in the context of a ripple-carry adder. This work was published in the IEEE ISCAS 2015.}
    } \\
  \end{tabularx}
\end{rSection}
